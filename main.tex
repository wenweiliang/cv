\documentclass[letterpaper,11pt]{cv}

% Show overfull hbox (black box at the end of line)
\setlength\overfullrule{5pt}

% Configure extra PDF metadata
\hypersetup{
    pdftitle={Wen-Wei Liang's CV},
    pdfauthor={Wen-Wei Liang},
    pdfsubject={},
    pdfcreator={LuaLaTeX}
}

% Override theme color
% Set theme color by RGB hex values
\definecolor{color-title}{HTML}{000000}

% Use custom fonts
\setmainfont{SourceSerif4}[
    Path=./fonts/source-serif/,
    Ligatures=TeX,
    UprightFont=*-Regular, ItalicFont=*-It,
    BoldFont=*-Bold, BoldItalicFont=*-BoldIt
]
\newfontfamily\lightsourceserif{SourceSerif4}[
    Path=./fonts/source-serif/,
    Ligatures=TeX,
    UprightFont=*-Light, ItalicFont=*-LightIt,
    BoldFont=*-Semibold, BoldItalicFont=*-SemiboldIt
]
\setsansfont{SourceSans3}[
    Path=./fonts/source-sans/,
    Ligatures=TeX,
    UprightFont=*-Regular, ItalicFont=*-It,
    BoldFont=*-Bold, BoldItalicFont=*-BoldIt
]
\newfontfamily\lightsourcesans{SourceSans3}[
    Path=./fonts/source-sans/,
    Ligatures=TeX,
    UprightFont=*-Light, ItalicFont=*-LightIt,
    BoldFont=*-Semibold, BoldItalicFont=*-SemiboldIt
]

% Define the bibliography path
\addbibresource{paperpile_annotated.bib}

% Style the custom notes (addendum in the biblatex entry) at the end of each reference
\DeclareFieldFormat{addendum}
    {\lightsourceserif\color{color-title}\bfseries\itshape (#1)}

% Style the volume (un-bold it)
\DeclareFieldFormat[article]{volume}{\normalfont #1}

% Force the publication title to be in sentence case
\DeclareFieldFormat[article]{titlecase}{\MakeSentenceCase*{#1}}

%%%%%%%%Begining
\begin{document}

% Heading

\begin{minipage}[b][][c]{35em}
    \lightsourceserif\Huge\bfseries\color{color-title}
    Wen-Wei Liang, Ph.D.
\end{minipage}%
\hfill
% Make sure there isn't any space between two minipages
% Ref: https://tex.stackexchange.com/a/114521
\hspace{1em}
\begin{minipage}[b][][b]{11em}
    \raggedleft
    \small
    101 6th Ave, New York, NY \hfill
    \href{mailto:wliang@nygenome.org}{wliang@nygenome.org} \hfill
    (314) 540-1364
\end{minipage}

\vspace{2ex}

% Tagline
{
    \small\lightsourceserif\bfseries\color{color-title}
    Molecular geneticist and genomics researcher specializing in CRISPR screening and multi-omic profiling, focusing on epigenetic regulation in tumorigenesis through integrated experimental and computational methods.\par
}

%%%%%%%%%%%Education
\section{Education}

\begin{entrylist}

\item \textbf{Washington University in St. Louis} \hspace{2.45em} Ph.D. in Molecular Genetics and Genomics \hfill
    2014--2020 \\

\item \textbf{National Yang-Ming University, Taiwan} \hspace{0.5em} M.S. in Microbiology and Immunology \hfill
    2009--2011 \\

\item \textbf{National Tsing-Hua University, Taiwan} \hspace{0.72em} B.S. in Life Science \hfill
    2005--2009 \\

\end{entrylist}

%%%%%%%Felloships
\section{Fellowships}

\begin{entrylist}

\item Taiwan Ministry of Education - Washington University in St. Louis Fellow (\$732,000) \hfill
    2014--2018 \\
\item Washington University in St. Louis Precision Medicine Pathway Fellow (\$10,000) \hfill
    2015--2016 \\
\item Taiwan Ministry of Education Fellowship to Study Abroad (\$10,000) \hfill
    2007--2008 \\

\end{entrylist}

%%%%%%%Research Experience
\section{Research Positions}

\begin{entrylist}
\raggedleft
\item \textbf{New York University \& New York Genome Center} \hfill New York, NY\\
Postdoctoral research associate, \href{http://sanjanalab.org/}{Laboratory of Neville Sanjana } \hfill
2021--Present
\begin{detaillist}
    \item \textit{\textbf{Transcriptome-scale RNA-targeting CRISPR screens reveal essential lncRNAs in human cells:}} Conducted Cas13 pooled screens and perturbation assays, coupled with single-cell RNA sequencing, to identify and characterize hundreds of lncRNAs crucial for cell proliferation and their mechanistic roles.
    \item \textit{\textbf{High-content CRISPR screens identify lncRNAs modulate CAR-macrophages in tumor microenvironment:}} Implemented \textit{in vitro} and \textit{in vivo} screens to discover lncRNAs that promote a pro-inflammatory microenvironment in BRCA xenograft and BRCA organoid models. 
\end{detaillist}

\item \textbf{Washington University in St. Louis} \hfill St. Louis, MO\\
Ph.D. student and postdoctoral research associate, \href{https://dinglab.wustl.edu/}{Laboratory of Li Ding } \hfill
2015--2021
\begin{detaillist}
    \item \textit{\textbf{Epigenetic regulation during cancer transitions across 11 tumour types:}} Utilized snATAC sequencing, single-cell RNA sequencing, and bulk whole-exome sequencing data from the Human Tumor Atlas Network (HTAN) to profile over a million cells or nuclei. Identified cancer-specific accessible region dynamics correlating with gene expression, uncovering novel regulatory regions and their target genes.
    \item \textit{\textbf{Integrative multi-omic cancer profiling reveals DNA methylation patterns associated with therapeutic vulnerability and cell-of-origin:}} Leveraged DNA methylation, RNA sequencing, and proteomic data from the Clinical Proteogenomic Tumor Analysis Consortium (CPTAC) to elucidate how epigenetic changes in cancer cells influence gene expression, protein levels, tumor characteristics, and therapeutic responses.
    \item \textit{\textbf{Driver fusions and their implications in the development and treatment of human cancers:}} Conducted comprehensive analysis of gene fusions in over 9,000 tumors across 33 cancer types from The Cancer Genome Atlas (TCGA), uncovering how these fusions drive the expression of oncogenes, tumor suppressor genes, and kinases. Identified druggable fusions in 6.0\% of cases, highlighting their potential as targets for precision therapies.
\end{detaillist}

\item \textbf{Academia Sinica} \hfill Taipei, Taiwan\\
Master's student and research assistant, Laboratory of Soo-Chen Cheng \hfill
2009--2014
\begin{detaillist}
    \item \textit{\textbf{A novel mechanism for Prp5 function in prespliceosome formation and proofreading the branch site sequence: }}Identified the role of Prp5 in spliceosome assembly and branch site proofreading, showing how Prp5's interactions with U2 snRNA and subsequent release facilitate accurate spliceosome assembly.
\end{detaillist}

\item \textbf{National Tsing-Hua University} \hfill Hsinchu, Taiwan\\
Undergraduate student, Laboratory of Chung-Yu Lan \hfill
2008--2009
\begin{detaillist}
    \item Investigated the reactions of the iron-responsive element under virulence gene expression in \textit{Candida albicans}.
\end{detaillist}

\item \textbf{Linköping University} \hfill Linköping, Sweden\\
Visiting student, Laboratory of Jordi Altimiras \hfill
2007--2008
\begin{detaillist}
    \item Analyzed G-protein coupled receptor-dependent contractility in chicken heart tissue.
\end{detaillist}

\end{entrylist}


\section{Publications}

\nocite{*}

\printbibliography[
    keyword={primary},
    heading=subbibliography,
    title={First authorship \footnotesize (* Equal contribution)}
    ]

\printbibliography[
    keyword={contributing},
    heading=subbibliography,
    title={Co-authorship}
    ]

\section{Honors \& Awards}

\begin{detaillist}

\item Best of Cell Reports 2018
    \hfill 2018

\item Outstanding Merit Research Award National Yang-Ming University
    \hfill 2011

\item Dean's List, Department of Life Science, National Tsing-Hua University
    \hfill 2007

\item Outstanding student from Low-income, Lu Feng-Zhang Memorial Scholarship
    \hfill
    2006

\end{detaillist}

\section{Selected Talks}
\begin{entrylist}
    \item \textbf{CSHL Meeting: The Biology of Genomes} \hfill Cold Spring Harbor, NY \\
    \textit{\small{Transcriptome-scale RNA-targeting CRISPR screens reveal essential lncRNAs in human cells}}
    \hfill 2024 

    \item \textbf{The CPTAC Annual Scientific Symposium} \hfill Virtual \\
    \textit{\small{Multiomic profiling reveals tumorigenic DNA methylation associated with therapeutic vulnerability and cell-of-origin}} \hfill 2021

    \item \textbf{The CPTAC Site Visit – NYU-WU-BYU PGDAC} \hfill New York University, NY \\
    \textit{\small{Tumorigenic DNA methylation revealed by integrative transcriptomic and proteomic profiling}} \hfill 2019

    \item \textbf{CSHL Meeting: Eukaryotic mRNA Processing} \hfill Cold Spring Harbor, NY \\
    \textit{\small{A novel mechanism for Prp5 function in prespliceosome formation and proofreading the branch site sequence}}
    \hfill 2013

    \item \textbf{Taiwan Yeast Meeting} \hfill Taiwan \\
    \textit{\small{The DEAD-box ATPase Prp5 mediates splicing fidelity control by counteracting tri-snRNP binding}}
    \hfill 2013
\end{entrylist}


\section{Teaching \& Mentoring}

\begin{entrylist}

\item \textbf{Simons foundation-NYU biology summer undergraduate research program, New York, NY} \hfill Summer 2024\\
    {\small Mentored an undergraduate student, Sebastián H. Díaz-Rodríguez, from a diverse background, providing high-quality research training and opportunities that are often inaccessible to underrepresented students.} \\

\item \textbf{Laboratory of Neville Sanjana, New York University, New York, NY} \hfill 2021--2024\\
    \begin{detaillist}
       \item \textbf{Breanna Williams}, Master's student, Recipient of the Wasserman Center Internship Grant\\
       \textit{Current position:} Research technician at Memorial Sloan Kettering Cancer Center
       \item \textbf{Olivia Choi}, Undergraduate, Recipient of the NYU Dean's Undergraduate Research Fund \\
       \textit{Current position:} Ph.D. student at John Hopkins University
    \end{detaillist}

\item \textbf{Laboratory of Li Ding, Washington University in St. Louis, St. Louis, MO} \hfill 2015--2020\\
    \begin{detaillist}
        \item \textbf{Carolyn Lou}, Undergraduate \\
        \textit{Current position:} Associate Director in Pfizer Biostatistics
        \item \textbf{Terrence Tsou}, Undergraduate \\
        \textit{Current position:} Medical student at John Hopkins University
        \item \textbf{Rita Jui-Hsien Lu}, Graduate research technician \\
        \textit{Current position:} Bioinformatician at Mount Sinai medical scool

    \end{detaillist}

\item \textbf{Department of Biology, Washington University in St. Louis, St. Louis, MO} \hfill Spring 2016\\
    {\small Delivered lectures, provided assistance in molecular biology experiments, and assessed student performance as a teaching assistance in the Microbiology Laboratory course (Biol 3491) for over 30 undergraduate students.}


\end{entrylist}


\section{Service \& Outreach}
\begin{entrylist}

    \item \textbf{Postdoc Seminar Series, New York Genome Center, New York, NY} \hfill 2021--Present \\
    {\small Organized monthly seminars featuring presentations by postdoctoral researchers.}

    \item \textbf{Biology Club, Rutgers University-Camden, Camden, NJ} \hfill 2022\\
    {\small Participated as a panelist to discuss career development opportunities in science and technology with over 20 undergraduate students.}
    
    \item \textbf{Midwest Taiwanese Biotechnology Association, Chicago, IL} \hfill 2017--2021\\
    {\small Co-founded the association, served in various leadership roles including Finance (2018), Promotion (2019), and President (2021), and organized events to foster dialogue among young scientists across Midwest cities.}

\end{entrylist}
\end{document}
